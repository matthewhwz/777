\subsection{Results}
The ideal sample results is summarized on the following table:
\\
\\
\begin{tabular}{p{5.25cm}p{5.25cm}p{6cm}}
\toprule
\textbf{Group size}&\textbf{Coordinate}&\textbf{Probability required}\\
\toprule
2&$(0.382,2)$&$p<0.382$\\
\midrule
3&$(0.343,3)$&$p<0.343$\\
\midrule
4&$(0.312,4)$&$p<0.312$\\
\bottomrule
\end{tabular}
\\
\\
\\
Observing the pattern, we found that the requirement of the infection rate $p$ increases as the groups size increases for the ideal solution.
\\
\\
The worse case results is summarized on the following table:
\\
\\
\begin{tabular}{p{5.25cm}p{5.25cm}p{6cm}}
\toprule
\textbf{Group size}&\textbf{Coordinate}&\textbf{Probability required}\\
\toprule
2&$(0.293,2)$&$p<0.293$\\
\midrule
3&$(0.306,3)$&$p<0.306$\\
\midrule
4&$(0.293,4)$&$p<0.293$\\
\bottomrule
\end{tabular}
\\
\\
\\
Observing the pattern, we found that the requirement of the infection rate $p$ is the same when sample size equal to 2 and 4 for the ideal solution. As we also calculated the range of $p$ for the worse case by program, from the data we get, we found that 3 have the highest comparability of range of $p$. Similar the ideal solution, the requirement of the infection rate $p$ increases as the groups size increases for the ideal solution after 4.\textbf{[Appendix 9.1]}
\\
\\
The difference results is summarized on the following table:
\\
\\
\begin{tabular}{p{3cm}p{3.5cm}p{3.5cm}p{6cm}}
\toprule
\textbf{Group size}&\textbf{Vertex}&\textbf{Inflection rate}&\textbf{Corresponding value of tests}\\
\toprule
2&$(0.5,0.25)$&0.5&0.25\\
\midrule
3&$(0.333,0.148)$&0.333&0.148\\
\midrule
4&$(0.25,0.105)$&0.25&0.105\\
\bottomrule
\end{tabular}
\\
\\
\\
Observing the pattern, it seems that the tests differ went smaller and smaller while the group size increases. This reasonable as the special case will always be 1 but the combinations of the group size is always increasing, the weight of the special case will become smaller and smaller as the group size increases. 