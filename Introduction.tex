\section{Introduction}
\subsection{Introduction}
During the pandemic of COVID-19(SARS-CoV-2), the demand for virus testing dramatically increases, so as the cost and wastes caused during the tests. By the traditional way, the subject is tested one by one, meaning that the amount of test conducted is as same as the number of subjects. When a huge amount of population need to be tested, overloading in laboratories will be leaded and a huge amount of waste will be generated. In this paper, we are going to investigate how to increase the efficiency of conducting test with less amount of test by pooling the samples together and doing multiple tests. We want to find a better solution for different conditions.
\subsection{Problem restatement}
\begin{enumerate}
  \item Develop a mathematical model for determining the most efficient testing method of COVID-19 for a large population. Discuss the factors taken into consideration, use the model to chose the most efficient way for testing (i.e. lower the total tests required).
  \item Finding the most efficient way of pooling test when the sample is grouped once
    \begin{enumerate}
     \item Choose an example from one of the cites around the Earth, and develop a mathematical model(or models) from any factors and data obtainable and we find significant for determining the optimum solution for the situation. Analyzing the result.
     \item Discuss any changes to the COVID-19 test model from that would be required to determine the result of the model.
    \end{enumerate}
  \item Discuss any changes to the COVID-19 test model that will be required when we consider regrouping.
  \item Compare different groupings effectiveness by the result.
\end{enumerate}